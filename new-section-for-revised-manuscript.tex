\section{Supporting students in navigating uncertainty with Bayes}

How can Bayes’ Theorem and the principles that follow from it help students to navigate the ubiquitous uncertainty in science? As long as science instruction progresses mostly qualitatively the features of Bayesian reasoning can help to bring a sense of coherence to the many ways in which uncertainty is manifest when students engage scientific practices such as constructing explanations, analyzing and interpreting data, or developing and using models. We argue that Bayesian reasoning can provide this coherence as its distinctive feature, that is, the “concept of degree of belief to describe epistemic attitudes about uncertain propositions” (Sprenger & Hartmann, 2019), embraces the epistemic stance that scientific knowledge is not immutable but in fact open to revision. In other words, scientific knowledge always comes attached with a degree of belief that can be updated following the rules of Bayes’ Theorem when new evidence becomes available. The result of that updating process depends on the uncertainty of existing scientific knowledge which is then expressed in the prior. In this way Bayesian reasoning emphasizes that science and scientific knowledge is always situated in context (Szu & Osborne, 2012) and society (Driver et al., 1994). These ideas can be captured in the following three statements:
 
a) Be skeptical of absolutes and preserve an open mind – scientific knowledge always comes with some uncertainty and priors that denoting absolute certainty or impossibility permit scientific progress. This first principle expresses the Bayesian degree of belief epistemology which reflects that scientific knowledge is tentative. 
b) Consider what you already know – do not evaluate data/information without considering prior evidence. This emphasizes how science is situated in context and society which both inform how evidence is evaluated.
c) Consider counterfactuals – consider evidence in terms of compatibility with all possible outcomes. This expresses the what in Bayesian philosophy is referred to as the Simple Principle of Conditionalization (Adams, 1965), i.e., when we weigh evidence we must consider to what extent it supports the range of possible explanations for the data. 

We argue that emphasizing these statements (or variations thereof) and how they relate to the nature of scientific knowledge can support students in connecting the frequent but often isolated references to uncertainty and limits of scientific knowledge in the scientific practices as expressed, e.g., in the Framework for K-12 Science Education (NRC, 2012). The descriptions of the practices in the Framework are filled with expressions like revising model, refining explanations, critiquing arguments, or considering the limitations of the precision of the data. However, what is lacking is an explicit rationale of why all this revising, refining, considering of limitations is necessary and in fact a feature of science, rather than a bug. Familiarizing students with the principles and core components of Bayesian reasoning can foster a habit of mind from which the elements of the practices that pertain to uncertainty and the limits of scientific knowledge follow naturally. This assumption is supported by the work of (Warren, 2018, 2020) which demonstrates that integrating Bayes reasoning into university science courses positively shifts students epistemic attitudes regarding the nature of knowing and learning. In the following we provide an example of how the quantitative Bayesian updating activities in (Warren, 2020) can be made accessible to students in middle and high school by using a computational tool. 

\section{Making it tangible with the confidence updater}

\subsection{The Confidence Updater Shiny app}

Warren (2020) frames the Bayesian reasoning as part of a hypothetico-deductive (HD) process (Popper, 1979). In the HD process, a model is evaluated by deriving a testable hypothesis and testing it against yet unknown data. Depending on ratio between the likelihood of observing the data if the hypothesis is true and the likelihood of observing the data if the hypothesis is false, confidence in the model decreases or increases relative to the initial confidence in the model. This confidence updating can be described using Bayes theorem, writing it in the following way: 

P(H|E) is confidence we can have in the hypothesis H after updating our initial confidence P(H) based on consideration of the evidence expressed in R. R > 1 represents confirmatory evidence, R = 1 represents inconclusive evidence, and 0 < R < 1 represents disconfirmatory evidence. Guidelines for choosing an adequate R exist (Kass & Raftery, 1995), e.g., 20 < R < 150 can be interpreted as the evidence strongly favoring H. P(H) the initial confidence in the hypothesis ranges from 0 to 1 with P(H) = 0.5 representing maximum epistemic uncertainty, i.e., having no idea about the validity of the hypothesis. To help students focus on the conceptual elements of updating their confidence in a hypothesis, the shiny confidence updater app in Figure XXX allows students to choose values for R based on their interpretation of the evidence and P(H) based on their initial confidence and after performing the needed calculations returns a textual statement about updated confidence P(H|E) in the Hypothesis after considering the evidence and an optional numeric value for the confidence level as well. 
 
Figure XXX. Shiny web application for updating one’s confidence in a hypothesis following Bayes Theorem. The orange P(H) and R are not displayed to students but imposed on the image to help orient the reader.

The principles a)-c) from the previous section are emphasized either implicitly or explicitly in the design of the confidence updater app. 
Principle a) “Be skeptical of absolutes and preserve an open mind” is emphasized as a natural consequence of Bayes Theorem and the options “absolutely certain that it is correct” and “absolutely certain that it is incorrect” in the “How sure are you about your hypothesis?” part of the confidence updater. When students select these options, P(H) is set to 1 or 0 respectively. In consequence, whatever updating factor R students choose, the updated confidence will always be the same as P(H) and the consideration of the evidence becomes pointless. Thus, if students selected any of the options that are equivalent to P(H) = 1, then observe strong contrary evidence, and wonder how this does not effect, the updated confidence are observing the evidence, a teacher could refer to principle a) and point out to students how fixed prior believes inhibit changes in believes even when confronted with strong evidence – as the students just experienced when using the confidence updater.
Principle b) “Consider what you already know” shows is emphasized in the wording of the prompt for selecting P(H) and becomes visible to students as it effectively moderates the power of the evidence. When prior believes and evidence align, the updated confidence is larger compared to either having “no idea” before or having prior believes that do not align. Students could notice this if they compare the updated confidence P(H|E) for students that had the same prior believe but evaluated the evidence differently or had different prior believes but evaluated the evidence in the same way. 

Finally, principle c) "Consider counterfactuals" is emphasized in how the prompt for R is worded. Further, when students use the confidence updater, students should be encouraged to argue for their choice of R and be able to explain why they choose a specific option. In such an argument, students following the prompt should not only consider how their evidence supports their own hypothesis but also how compatible it is with alternative hypotheses. 

\subsection{An example}

How does using the confidence updater change a typical science class activity? Here we sketch how using the confidence updater can provide rich opportunities to discuss and reflect on epistemic attitudes. A standard activity in German middle school is to figure out the relation between electric current I, resistance RE, and voltage U, i.e., Ohm’s law (U=RE*I). To do so, students one can measure the resulting current for different voltages in a circuit with fixed resistance. At the beginning of the activity, students are asked to come up with different hypotheses for the relationship. They often propose a proportional, inverse, or quadratic relationship. Now, the students may form groups, select a hypothesis, and set out to test it. Following principle a) students can now specify their prior confidence in their group’s hypothesis. As the prior is subjective students may very well choose different priors based on their individual experiences. Now, they may proceed and record a number of measurements of the current for different voltages and then graph the data. Figure XXX shows an example graph. Now, when students are asked to evaluate the Evidence and set a value for R, they are urged to consider to what extent the evidence is consistent with their hypothesis vs. an alternative which is emphasized in principle c). Looking at the graph one could see evidence for a proportional relationship as well as a higher order positive relationship (blue and red dashed line in Figure XXX). Thus, the evidence is barely compatible with an inverse relationship between I and U but supportive of both hypotheses that suggest a positive relationship between the variables. Thus, depending on what hypothesis students choose to investigate, they should select “data favors an alternative hypothesis” for R if their hypothesis is inverse relationship and select “data somewhat favors my hypothesis” if their hypothesis is proportional or quadratic relationship. Now, the updated confidence for every student will depend on their initial confidence in their hypothesis and the evaluation of the data which will be different across groups. If students compare their results within groups, they should see how the effect of their initial confidence influences the effect of the evidence. However, if they repeat the activity, i.e., use their updated confidence as a new initial believe and collect more data, they should see how consensus is approached over a few iterations unless anyone choose P(H) = 0 or P(H) = 1.  Frist, this emphasizes the importance of principle a) for science again. Second, as (Warren, 2018) points out, “the fact that everyone starts with subjective prior probabilities and yet inevitably converges to a single asymptotic result illustrates the objective aspect of science” which can help build trust in science. Lastly, if student compare their results between the groups, students can note how disconfirmatory evidence has more epistemic power than confirmatory evidence, which is also another direct corollary of Bayes Theorem.

\subsection{Figure XXX. Possible graph of current and voltage data.}
