
\section{How Bayesian Principles Could Bolster Science Literacy}

We conceive that Bayesian methods can enhance \ital{science literacy}, capabilities range from core \ital{literacies}, including numeracy and textual literacy (what is mostly thought of in common parlance concerning literacy), content knowledge, and epistemic knowledge \parencite{council2016science}. The specific ways we envision a Bayesian perspective bolstering science literacy are also techniques that teachers can appropriate in their planning and teaching; they may already align with how they teach about science through engaging in scientific practices \parencite{nrc12} or how they articulate the nature of science \parencite{lederman2020avoiding}. \\

Consider a salient topic as a context to apply these principles, medical vaccines for diseases, such as the disease caused by the coronavirus SARS-CoV-2, COVID-19. Most evidence suggests that vaccines have been safe, historically, and are generally safe, at present (Centers for Disease Control and Prevention, 2021). Nevertheless, they are not perfectly safe, as unlikely, adverse outcomes do happen, and there may be adverse side effects from some vaccines that research has not yet revealed. In light of this evidence, consider the development of a new vaccine, such as those available at the time of this writing for COVID-19. Some use long-established methods of vaccine development (namely, deactivating SARS-CoV-2 particles but retaining cell signaling proteins to which humans develop an immune response—and, after, a degree of immunity) and injecting these deactivated virus particles into people), whereas others use a new mechanism (namely, injecting messenger RNA into people; this messenger RNA is then transcribed by cells in people’s bodies to produce a particular protein common to SARS-CoV-2 particles, which human’s bodies then learn to recognize to develop immunity. We note that--at present--diseases for which it has been difficult to develop effective vaccines, including Malaria, may have viable mRNA vaccines in the near future \parencite{datoo2021high}. \\

Before any data from clinical trials, it would be reasonable to consider these vaccines to probably be safe in light of the safety of other vaccines (and based on similarities in their mechanisms of action), but to have a strong degree of caution about how safe they are. After initial trials were completed, the degree of safety that individuals associate with these vaccines—in light of the prior and the limited evidence—could increase, but still not be near the level of safety that many would consider acceptable. Following the completion of clinical trials that show minimal side effects (holding aside their efficacy), one’s estimates of the safety of these vaccines could increase by a substantial amount, but concerns could remain, particularly for messenger RNA vaccines, which are newer, and may, plausibly, have unintended side effects. \\

While the weighting of the evidence in light of prior research and theoretical accounts above may be natural to a scientist, it is challenging for many. In the following, we showcase how several important insights for weighting evidence follow directly from Bayes' rule. We do not think or expect that members of the public can do this, but members of the public (and learners) do have viable ideas and hypotheses about how the world works. Thus, we argue that emphasizing Bayesian reasoning can support science literacy by providing people with a principled way to weight evidence that is accessible because it builds on peoples' prior ideas. \\

In the context of the above example on SARS-CoV-2, the result of applying Bayes Theorem in a heuristic mode is a non-binary inference, one that doesn't speak to whether or not the vaccine is safe, but to what degree it is safe, doing so with a language of probability and uncertainty. Though a small difference, discussing the safety of vaccines on these terms can provide a more solid foundation for principled disagreements and arguments. In addition, Bayes provides a heuristic and a language for engaging with uncertainty in a science literacy context in a principled way that is in line with how most of us likely already interpret data \parencite{tgk06, gw12, gh95}; A Bayesian perspective can provide people with a way to talk about weighing between ideas and their prior beliefs and new information and data. In short, Bayes gives people a chance to deal with uncertainty in a principled but accessible manner. \\

In addition, many themes introduced in the section on the epistemological principles that follow from Bayes Theorem are relevant to the aim of supporting science literacy. First, \ital{even individuals with different initial beliefs may be brought closer to an agreement in light of data} that serves as evidence for a particular belief or hypothesis. In this way, Bayes Theorem illustrates--at the conceptual level as well as statistically--how individuals may be unlikely to agree based on the limited or weak information, but that this can shift in light of accumulating data. As we noted earlier, this is far from a deterministic process, and \ital{holding initial beliefs that a hypothesis has a probability of zero or one means that data does not affect one's beliefs}. This suggests that a cultural norm and principle for science literacy can be to always reserve a degree of uncertainty such that one could be influenced by (new) data or evidence: From a Bayesian perspective, \ital{our knowledge is uncertain and open to change and revision} so long as we remain open to new evidence. \\

Related to the importance of not holding scientific beliefs with a probability of zero or one, another principle that follows from Bayes Theorem is that \ital{extraordinary claims require extraordinary evidence}, and that unlikely or implausible hypotheses or claims are likely to necessitate strong empirical data to be convincing to others. A corollary of this principle is that we should strong initial or prior beliefs can serve as a tool through which we can discuss extraordinary claims: Except in light of strong evidence, "ordinary" claims can serve as the default for citizens and scientists making sense of unusual and potentially biased claims. Similarly, Bayes Theorem demonstrates that the \ital{simplest possible explanation--the most parsimonious explanation--is often the best explanation}. \\ 

In sum, a Bayesian perspective offers a general approach to science literacy that can account for uncertainty in a rigorous and principled manner. In addition, a number of principles that follow from Bayes Theorem, such as being receptive to new evidence and weighting that evidence per its strength in light of the strength of one's initial beliefs or understanding could bolster science literacy--and how the public comes to understand and gain trust in science. \\